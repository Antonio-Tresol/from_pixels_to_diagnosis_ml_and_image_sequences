% Commented as requested by the teacher
% \subsubsection{Segmentation of 2D images}

% Commented to make it more concise
% The segmentation and classification of intracranial hemorrhage (ICH) using CT imaging has been the subject of extensive research, leading to various methodologies aimed at improving diagnostic accuracy.

A significant number of studies have focused on the segmentation of hemorrhagic lesions. This serves as a foundational step toward volume estimation and outcome prediction. For instance, Yu et al. \cite{Yu2022} demonstrates effective segmentation techniques using a UNet for identifying ICH in brain CT scans. Similarly, Yao et al. \cite{YAO2020101910} present an automated approach that enhances the segmentation process, highlighting the importance of accurate delineation in improving clinical outcomes.

% Commented as requested by the teacher
% \subsubsection{Segmentation with 3D images}

In terms of advanced imaging techniques leveraging multiple images (3D images or sequences), Lu et al. \cite{9210782} provide a compelling argument for the application of data augmentation strategies by mirroring the original CT image, which can enhance model performance. Furthermore, models like the improved 3D U-Net and those utilizing squeeze-and-excitation blocks have demonstrated promising results in achieving precise segmentation and volume measurements \cite{ABRAMOVA2021101908}.

% Commented as requested by the teacher
% \subsubsection{Classification of 2D images}

Conversely, the classification of intracerebral and intracranial hemorrhages have been explored through various approaches. For example, Thabarsa et al. \cite{Thabarsa2023299} illustrate how support vector machines (SVM) have been effectively utilized to classify intracerebral hemorrhages based on radiomic features. Additionally, Seymur et al. \cite{Seymour2022} showcase the application of multiple machine learning techniques, including random forests, K-nearest neighbors (KNN), support vector machines, multilayer preceptrons, naive bayes and logistic regression for predicting outcomes related to intracranial hematoma expansion.

Recent advances in deep learning have also enabled more robust approaches for ICH classification. The study of Zhong et al. \cite{Zhong2021} exemplifies the successful use of convolutional neural networks (CNNs) for this purpose. Moreover, Ma et al. \cite{Ma2022} further emphasize the capability of CNNs to predict hematoma expansion accurately.

% Commented as requested by the teacher
% \subsubsection{Classification of 3D images}

Despite the progress made in classification using single images, there is a scarcity of studies focusing on the classification of ICH with multiple CT slices. The work done by Grewal et al. \cite{8363574}, highlights the potential of utilizing multiple images in achieving higher accuracy, primarily through recurrent neural networks such as Long Short-Term Memory (LSTM) models. In a similar vein, Zhou et al. \cite{Zhou2024}, demonstrate the application of both 3D CNNs and Swin Transformers, as well as remarking the benefits of multi-view approaches in medical image analysis.


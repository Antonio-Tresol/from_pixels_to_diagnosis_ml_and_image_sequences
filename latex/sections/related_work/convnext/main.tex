In medical image classification deep learning has been used to try to speed up and automate the process of detecting a condition based on a patients medical images as wells as tasks like image segmentation \cite{dlmedicalimages}. Architectures like Convolutional Neural Networks and Transformers have played an important role in recent years as well as techniques such as transfer learning \cite{dlmedicalimages}.

In this context, ConvNeXT has emerged as a state-of-the-art model for image classification \cite{convnextbro}. Its applications in medical imaging are particularly noteworthy, especially within the context of brain imaging.  For instance, Nizamli et al. \cite{nizamli_accurate_2024} achieved up to 95.44\% accuracy using ConvNeXT for brain tumor detection in CT and MRI scans. Similarly, Panthakkan et al. \cite{panthakkan_unleashing_2023} used ConvNeXT to classify brain tumors with 99\% accuracy.  Furthermore, Sharma et al. \cite{sharma_neurospectra_2023}  demonstrated its effectiveness in multiclass classification of Alzheimer's disease from brain MRI images, achieving an accuracy of 98.89\%. Notably, all these studies utilized pretrained weights on ImageNet 1k for transfer learning. 

Beyond brain imaging, ConvNeXT has shown promise in classifying images related to other conditions, such as diabetic retinopathy \cite{devanshi_early_2023} and breast cancer \cite{reddy_enhancing_2024} where it stands out the high effectiveness given that breast cancer screening produces a sequence of images and individual images were use to classify as positive or negative.
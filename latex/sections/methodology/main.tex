% Work in progress
% \begin{itemize}
%     \item Mencionar parámetros e hiperparámetros del entrenamiento (épocas, optimizador, learning rate, calendarizador, early stopping etc)
%     \item Mencionar origen y descripción de los datasets
%     \item Mencionar uso de los datasets
%     \item Especificaciones del hardware utilizado
%     \item Mencionar modelo multiview y por qué fue seleccionado
%     \item Mencionar modelo de video y por qué fue seleccionado
%     \item a) entrenar los modelos sobre los datos de entrenamiento. Se guarda el best model en las épocas de entrenamiento.
%     \item b) Luego de entrenados, se toman las métricas (cuál/es). 
%     \item a) y b) se repiten $n$ veces, generando $n$ observaciones por modelo.
%     \item descripción de pruebas estadísticas a realizar (breve)
% \end{itemize}

In this section we provide a detailed explanation of the approach taken to collect, analyze, and interpret data. This section is composed of the dataset description and preprocessing methods, models, metrics, experiments descriptions and the statistical analysis.

\subsection{Dataset}
For the video model, we selected a transformer-based architecture: ViVit \cite{DBLP:journals/corr/abs-2103-15691}. This choice was motivated by the notable performance of transformers in single-image classification, the ability of these architectures to capture long-term dependencies through attention mechanisms, and the demonstrated effectiveness of ViVit as a modern, accessible model achieving promising results \cite{DBLP:journals/corr/abs-2103-15691}.

The specific ViVit model being used was Google’s Vivit-b-16x2-kinetics400, which features a base backbone comprising 12 transformer layers, each with a 12-head self-attention block \cite{DBLP:journals/corr/abs-2103-15691}. ViVit captures spatio-temporal information by structuring data into `tubes' across three dimensions: height, width, and temporal, which in this case were set to 16, 16, and 2, respectively \cite{DBLP:journals/corr/abs-2103-15691}. This model was pretrained on the Kinetics-400 dataset and only the weights of the final layer were modified during training.

% Commented as requested by the teacher
% \begin{figure}[H]
%     \centering
%     \includegraphics[width=0.8\linewidth]{imgs//models/vivit_tubes.png}
%     \caption{ViVit `tubes' representation \cite{DBLP:journals/corr/abs-2103-15691}.}
%     \label{fig:vivit_tubes}
% \end{figure}

% Commented to make it more concise
% ViVit required additional data handling to transform the images into a readable structure. This involved converting each patient's sequence of slices into a tensor, which was then added to a new dataset along with its corresponding label (positive or negative). The dataset was subsequently saved as a file, allowing for reuse and eliminating the need to repeat this computationally intensive step.

The model was used with the following parameters, which yielded optimal performance: a validation proportion of 0.3, a learning rate of 0.000005, and a maximum of 20 epochs with early stopping to prevent overfitting.

\subsection{Models}
For the video model, we selected a transformer-based architecture: ViVit \cite{DBLP:journals/corr/abs-2103-15691}. This choice was motivated by the notable performance of transformers in single-image classification, the ability of these architectures to capture long-term dependencies through attention mechanisms, and the demonstrated effectiveness of ViVit as a modern, accessible model achieving promising results \cite{DBLP:journals/corr/abs-2103-15691}.

The specific ViVit model being used was Google’s Vivit-b-16x2-kinetics400, which features a base backbone comprising 12 transformer layers, each with a 12-head self-attention block \cite{DBLP:journals/corr/abs-2103-15691}. ViVit captures spatio-temporal information by structuring data into `tubes' across three dimensions: height, width, and temporal, which in this case were set to 16, 16, and 2, respectively \cite{DBLP:journals/corr/abs-2103-15691}. This model was pretrained on the Kinetics-400 dataset and only the weights of the final layer were modified during training.

% Commented as requested by the teacher
% \begin{figure}[H]
%     \centering
%     \includegraphics[width=0.8\linewidth]{imgs//models/vivit_tubes.png}
%     \caption{ViVit `tubes' representation \cite{DBLP:journals/corr/abs-2103-15691}.}
%     \label{fig:vivit_tubes}
% \end{figure}

% Commented to make it more concise
% ViVit required additional data handling to transform the images into a readable structure. This involved converting each patient's sequence of slices into a tensor, which was then added to a new dataset along with its corresponding label (positive or negative). The dataset was subsequently saved as a file, allowing for reuse and eliminating the need to repeat this computationally intensive step.

The model was used with the following parameters, which yielded optimal performance: a validation proportion of 0.3, a learning rate of 0.000005, and a maximum of 20 epochs with early stopping to prevent overfitting.

\subsection{Metrics}
For the video model, we selected a transformer-based architecture: ViVit \cite{DBLP:journals/corr/abs-2103-15691}. This choice was motivated by the notable performance of transformers in single-image classification, the ability of these architectures to capture long-term dependencies through attention mechanisms, and the demonstrated effectiveness of ViVit as a modern, accessible model achieving promising results \cite{DBLP:journals/corr/abs-2103-15691}.

The specific ViVit model being used was Google’s Vivit-b-16x2-kinetics400, which features a base backbone comprising 12 transformer layers, each with a 12-head self-attention block \cite{DBLP:journals/corr/abs-2103-15691}. ViVit captures spatio-temporal information by structuring data into `tubes' across three dimensions: height, width, and temporal, which in this case were set to 16, 16, and 2, respectively \cite{DBLP:journals/corr/abs-2103-15691}. This model was pretrained on the Kinetics-400 dataset and only the weights of the final layer were modified during training.

% Commented as requested by the teacher
% \begin{figure}[H]
%     \centering
%     \includegraphics[width=0.8\linewidth]{imgs//models/vivit_tubes.png}
%     \caption{ViVit `tubes' representation \cite{DBLP:journals/corr/abs-2103-15691}.}
%     \label{fig:vivit_tubes}
% \end{figure}

% Commented to make it more concise
% ViVit required additional data handling to transform the images into a readable structure. This involved converting each patient's sequence of slices into a tensor, which was then added to a new dataset along with its corresponding label (positive or negative). The dataset was subsequently saved as a file, allowing for reuse and eliminating the need to repeat this computationally intensive step.

The model was used with the following parameters, which yielded optimal performance: a validation proportion of 0.3, a learning rate of 0.000005, and a maximum of 20 epochs with early stopping to prevent overfitting.

\subsection{Experiments}
For the video model, we selected a transformer-based architecture: ViVit \cite{DBLP:journals/corr/abs-2103-15691}. This choice was motivated by the notable performance of transformers in single-image classification, the ability of these architectures to capture long-term dependencies through attention mechanisms, and the demonstrated effectiveness of ViVit as a modern, accessible model achieving promising results \cite{DBLP:journals/corr/abs-2103-15691}.

The specific ViVit model being used was Google’s Vivit-b-16x2-kinetics400, which features a base backbone comprising 12 transformer layers, each with a 12-head self-attention block \cite{DBLP:journals/corr/abs-2103-15691}. ViVit captures spatio-temporal information by structuring data into `tubes' across three dimensions: height, width, and temporal, which in this case were set to 16, 16, and 2, respectively \cite{DBLP:journals/corr/abs-2103-15691}. This model was pretrained on the Kinetics-400 dataset and only the weights of the final layer were modified during training.

% Commented as requested by the teacher
% \begin{figure}[H]
%     \centering
%     \includegraphics[width=0.8\linewidth]{imgs//models/vivit_tubes.png}
%     \caption{ViVit `tubes' representation \cite{DBLP:journals/corr/abs-2103-15691}.}
%     \label{fig:vivit_tubes}
% \end{figure}

% Commented to make it more concise
% ViVit required additional data handling to transform the images into a readable structure. This involved converting each patient's sequence of slices into a tensor, which was then added to a new dataset along with its corresponding label (positive or negative). The dataset was subsequently saved as a file, allowing for reuse and eliminating the need to repeat this computationally intensive step.

The model was used with the following parameters, which yielded optimal performance: a validation proportion of 0.3, a learning rate of 0.000005, and a maximum of 20 epochs with early stopping to prevent overfitting.

\subsection{Statistical analysis}
For the video model, we selected a transformer-based architecture: ViVit \cite{DBLP:journals/corr/abs-2103-15691}. This choice was motivated by the notable performance of transformers in single-image classification, the ability of these architectures to capture long-term dependencies through attention mechanisms, and the demonstrated effectiveness of ViVit as a modern, accessible model achieving promising results \cite{DBLP:journals/corr/abs-2103-15691}.

The specific ViVit model being used was Google’s Vivit-b-16x2-kinetics400, which features a base backbone comprising 12 transformer layers, each with a 12-head self-attention block \cite{DBLP:journals/corr/abs-2103-15691}. ViVit captures spatio-temporal information by structuring data into `tubes' across three dimensions: height, width, and temporal, which in this case were set to 16, 16, and 2, respectively \cite{DBLP:journals/corr/abs-2103-15691}. This model was pretrained on the Kinetics-400 dataset and only the weights of the final layer were modified during training.

% Commented as requested by the teacher
% \begin{figure}[H]
%     \centering
%     \includegraphics[width=0.8\linewidth]{imgs//models/vivit_tubes.png}
%     \caption{ViVit `tubes' representation \cite{DBLP:journals/corr/abs-2103-15691}.}
%     \label{fig:vivit_tubes}
% \end{figure}

% Commented to make it more concise
% ViVit required additional data handling to transform the images into a readable structure. This involved converting each patient's sequence of slices into a tensor, which was then added to a new dataset along with its corresponding label (positive or negative). The dataset was subsequently saved as a file, allowing for reuse and eliminating the need to repeat this computationally intensive step.

The model was used with the following parameters, which yielded optimal performance: a validation proportion of 0.3, a learning rate of 0.000005, and a maximum of 20 epochs with early stopping to prevent overfitting.

% Having outlined the methodology, we now turn to the analysis of the results obtained from its application in the experiment.
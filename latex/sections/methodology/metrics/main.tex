The following metrics were used to evaluate the models efficacy:

\subsubsection{Confusion Matrix}

A confusion matrix is a table used to evaluate the performance of a classification model by comparing actual target values with the model’s predictions. It summarizes outcomes as true positives (TP), where ICH is correctly predicted; false positives (FP), where ICH is predicted but not present; true negatives (TN), where ICH is correctly predicted as absent; and false negatives (FN), where ICH is present but not predicted.
% Commented to make it more concise
% Using the values from the confusion matrix, several performance metrics can be calculated, such as the accuracy, precision and recall.

\subsubsection{Accuracy}
It represents the ratio of correctly predicted instances to the total number of instances in the dataset. Mathematically, accuracy can be defined as:

\begin{equation*}
    \text{Accuracy} = \frac{TP + TN}{TP + TN + FP + FN}
\end{equation*}

\subsubsection{Precision}
It measures the accuracy of positive predictions made by the model, indicating the proportion of instances classified as positive that are actually positive. Mathematically, precision can be defined as:

\begin{equation*}
    \text{Precision} = \frac{TP}{TP + FP}
\end{equation*}

\subsubsection{Recall}
Recall, also known as sensitivity or true positive rate, is a metric used to evaluate the performance in scenarios where the focus is on correctly identifying positive instances. It calculates the proportion of actual positive instances that were correctly identified by the model. Mathematically, recall can be defined as:

\begin{equation*}
    \text{Recall} = \frac{TP}{TP + FN}
\end{equation*}

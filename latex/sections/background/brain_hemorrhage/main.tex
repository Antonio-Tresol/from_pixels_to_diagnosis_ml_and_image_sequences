% Commented as requested by the teacher
% \subsubsection{Intracraneal hemorrhage}

Intracranial hemorrhage (ICH) is a type of hemorrhage that occurs in specific parts of the brain and is usually caused by accidents that involve brain lesions \cite{brain-dataset}. This condition can lead to increased intracranial pressure, which may cause severe neurological deficits, brain herniation, infarcts, rebleeds, vasospasms, seizures, and even death \cite{tenny_intracranial_2024}. Rapid diagnosis and intervention are crucial for improving patient outcomes, as delayed treatment can result in irreversible brain damage or complications.

% This condition is normally divided into four types: Epidural hemorrhage, subdural hemorrhage, subarachnoid hemorrhage (SAH) and intraparenchymal hemorrhage \cite{tenny_intracranial_2024}. This separation is based upon the hemorrhage location in the brain \cite{brain-dataset}. Nevertheless, this paper focuses on a binary classification of ICH, regardless of its type.

It is also important to clarify the terminology used in the literature regarding intracranial hemorrhage (ICH) and hemorrhage in general. While some studies specifically refer to ICH as synonymous with intracerebral hemorrhage \cite{Nag2023, Thabarsa2023299}, others use it to denote the broader concept of intracranial hemorrhage as described by Seymour et al. \cite{Seymour2022}.

%Finally, a great variety of literature use the term hematoma instead of referring to ICH. The term \quotes{hematoma} is frequently employed to describe fluid-filled masses that contain varying amounts of blood \cite{TVEDTEN2012337}. The underlying reason for this term exchange is the close relation to ICH, since it is a possible indication of it.

% Commented as requested by the teacher
% \subsubsection{CT Scans}

Independently of the type of hemorrhage prognosis being studied, non-contrast computed tomography (CT) scans of the head are one of the most common imaging modalities to evaluate the patient's condition \cite{tenny_intracranial_2024}. This is a cross-sectional imaging technique that offers extra diagnostic insights in cases where standard radiography is inadequate. Multiple images (called slices) are obtained from a single scan. They are created by an X-ray beam that rotates in a circular gantry around the object. 

% Commented to be more concise
% The gantry and the table supporting the patient move in a carefully coordinated manner, making a series of precise increments \cite{MACKAY2012676}.

These images can form a 3D image, being the third dimension the space and time. The slices represent the movement of the gantry across the patient. As a consequence, some studies refer to CT Slices as 3D images.


Transfer learning is a way to translate a model's learned patterns to approximate another problem of the same type or a very similar type compared to the original problem. \cite{yosinski2014transferablefeaturesdeepneural} This technique is often used to solve problems that may not have a lot of direct observations, but share similarities with problems that have already been solved using machine learning techniques \cite{yosinski2014transferablefeaturesdeepneural}. Another common use is when the resources to train the model are scarce or insufficient to achieve remarkable results \cite{yosinski2014transferablefeaturesdeepneural}.

A problem of brain hemorrhage classification, like the one that concerns us in this paper, is often difficult to approach due to training data availability \cite{zhuang_2021_a}. This challenge renders transfer learning an ideal approach for addressing the problem. In hindsight, most of the previous work on classification and segmentation concerning medical datasets exhibits a heavy reliance on transfer learning methods \cite{zhuang_2021_a}.